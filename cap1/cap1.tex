
%-------------------------------------------------------------
%				INTRODUZIONE
%-------------------------------------------------------------

\mcchap{Introduzione}{cap:intro}

Nel periodo di apprendistato\cite{ETICHETTA} in 7Pixel sono stato assegnato al team Iguana, team che si occupa
della gestione principalmente front-end dei siti del \emph{Kirivo Network}

\section{Il Kirivo Network}
Il Kirivo Network (KN) è attualmente composto da due siti  \url{www.kirivo.it} 
e \url{www.origini.it}:

\url{www.kirivo.it} è un negozio online che vende prodotti di tutte le categorie.
Il marketplace dispone di un offerta di oltre 800.000 articoli in tutte le categorie tra cui
elettrodomestici, prodotti per la casa, smartphone e TV, giocattoli, moda e altri.

\url{www.origini.it} è una divisione verticale di Kirivo. Il sito è specializzato nella vendita
di vini e offre un ampia offerta di prodotti divisi per cantine e regioni. Il sito
è online da Novembre 2016.

I siti del Kirivo Network fanno utilizzo di servizi di Back-End comuni che permettono 
di effettuare acquisti nei due siti utilizzando un unico account ed un unico carrello.
%Con buone probabilità verranno aggiunti in futuro nuovi siti verticali per alcune categorie
%di Kirivo.

Le applicazioni del Kirivo network sono serviti tramite un web server \emph{Ruby} implementato
usando il framework \emph{Ruby on Rails} chiamato \emph{Kiruby}

\section{Architettura del Kirivo Network}
Per l'erogazione dei siti del Kirivo network vengono usati diversi server
\begin{itemize}
\item {\bf Hybris}: una piattaforma Enterprise di e-commerce scritta in Java che offre una soluzione all-in-one per i siti 
di e-commerce comprendendo servizi quali la gestione del catalogo dei prodotti, degli utenti e la 
gestione sicura dei pagamenti.

Hybris utilizza una database relazione Postgres ed utilizza il motore di ricerca Java SolR.

Il sito Kirivo è nato utilizzando una piattaforma a pagamento per velocizzare i tempi di sviluppo.
\item {\bf Kitty}: un server di backend in Ruby, attualmente si sta lavorando nel migrare gradualmente tutte le funzionalità
di Hybris in Kitty in modo da non aver più bisogno in futuro della piattaforma di e-commerce.
\item {\bf Kiruby}: un web-server Ruby che eroga le pagine web dei siti, si interfaccia con Hybris e Kitty utilizzando i loro
servizi di backend. 
\item Wordpress: usato per la creazione di pagine di contenuto che vengono incluse da Kiruby.

Il server di Wordpress si trova nella LAN aziendale e si interfaccia solamente con il server Kiruby.
\item {\bf Redis}: un database noSql che, salvando tutto il suo contenuto in RAM, garantisce alte prestazioni
e viene usato da Kiruby come cache di contenuti. 
\item {\bf Nginx}: un Reverse proxy usato per redirigere le chiamate ai server opportuni.
\end{itemize}

\section{I team del Kirivo Network}
Lo sviluppo è l manutenzione dei siti del Kirivo Network viene effettuato da più team e questi sono:
\begin{itemize}
\item {\bf Team Iguana}: si occupano dello sviluppo di Kiruby.
\item {\bf Team Nimbus}: si occupano dello svlippo di Hybris e Kitty.
\item {\bf Content Manager}: si occupano della gestione dei prodotti e della creazione di pagine di speciali 
utilizzando Hybris e Wordpress
\item {\bf Grafica}: lavora o direttamente su Kiruby o da al team Iguana grafiche Html che vengono poi integrate con i 
vari servizi.
\end{itemize}

\section{Gestione delle homepage} 	

Le homepage di Origini e Kirivo sono le pagine che, nei rispettivi siti, possono cambiare contenuto più 
frequentemente. Inoltre scegliere quali prodotti, quali offerte e quali contenuti 
vanno inseriti in homepage non è compito dei programmatori ma dei \emph{content},
quindi si rivela importante dare la possibilità ai content di fare modifiche, come cambiare un prodotto da
mettere tra quelli in evidenza, senza dover passare ogni volta dai programmatori.

Per dare più libertà ai content, il contenuto della homepage, ovvero tutto quello che non è
header e footer, non si trova nel server Kiruby ma in pagine Wordpress che i content possono direttamente 
modificare accedendo alla sezione admin di Wordpress.

Il server Kiruby quando deve visualizzare la home chiede a Wordpress la pagina della home, ne estrae
il contenuto e lo renderizza nell'html tra Header e Footer

\section{La gemma cmsdealer}
Per visualizzare contenuti dinamici, come ad esempio un Box di 4 vini viene utilizzata
una gemma Ruby creata da un collega chiamata \emph{cmsdealer}.

Questa gemma viene usata dal server Kiruby. Questo mentre scansiona la pagina di Wordpress da includere
se incontra un tag di nome \emph{cmsdealer} legge l'attributo \emph{type} e in base al valore di questo
seleziona il corrispondente template, legge l'ID dei prodotti e stampa l'html del box con
i valori dei prodotti selezionati

Esempio: se nella pagina Wordpress Kiruby trova
\begin{verbatim}
	<dynamic type="OriginiListingBox" ids="3422,2345,2872,2209" />
\end{verbatim}
allora verrà cercato il template di \emph{originilistingbox.html.erb} e verrà popolato
coi valori dei prodotti con gli id specificati nell attributo \emph{ids}.


\newpage
\section{Obbiettivo}
L'obbiettivo del progetto è quello di rendere l'edit delle pagine
da parte dei content molto più semplice e flessibile,
facendo in modo che i contenuti delle homepage possano essere editati visualmente e non
andando a mettere mano direttamente sul codice html.

Per farlo i content dovranno interagire con un interfaccia grafica web che permette
la modifica delle informazioni necessarie e di poter spostare e copiare
componenti della home con un click o con un drag and drop.

\section{Obbiettivo secondario}
Obbiettivo secondario del progetto e di rendere il sito più manutenibile.

La criticità del sistema è che contiene codice duplicato nella gemma CmsDealer, in Kiruby
e nei widget di Wordpress. 

Per risolvere il problema sono state create le  \emph{RenderdPricesAPI} 
che restituiscono frammenti di html renderizzato per varie componenti della home.

In questo modo il codice del template resta solamente in Kiruby e esponendo le 
API questo viene usato sia dai Widget di Wordpress sia dalla gemma CmsDealer

