%%%%%%%%%% CAPITOLO DI TESI %%%%%%%%%%
%
% Capitolo "1" Introduzione
%
%%%%%%%%%%%%%%%%%%%%%%%%%%%%%%%%%%%%%%
\mcchap{Introduzione}{cap:intro}

\section{Introduzione}
%% commenti
Questa è l'introduzione.\\ 
%% a capo

\subsection{è una sotto sezione}
Fai dei punti:
\begin{itemize}
\item punto 1
\item punto 2
\item punto 3
\end{itemize}

\subsection{code}
\lstinputlisting[style=customjava, language=Java]{code/HelloWorld.java}

\subsection{table}

\begin {table}[H]
\caption {Tabella di prova} \label{tab:tabprova} 
\begin{center}
\begin{tabular}{|c|c|c|}
  
  \hline
  \rowcolor[gray]{.6}
  \textbf{colonna1} & \textbf{colonna2} & \textbf{colonna3} \\
  
  \hline
  \rowcolor[gray]{.8}
  riga1 &  5 & 15\\
  
  \hline
  \rowcolor[gray]{.9}
  riga2 &  3 & 3\\
  
  \hline
  \rowcolor[gray]{.8}
  riga3 & 5 & 29.5\\
  
  \hline
\end{tabular} 
\end{center}
\end{table}

\subsection{image}

\begin{figure}[htbp]
\begin{flushright}
\centering
\includegraphics[width=0.25\textwidth]{figure/tex.png}
\caption{Immagine di Latex}
\label{fig:gc}
\end{flushright}
\end{figure}
