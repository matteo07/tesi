
%-------------------------------------------------------------
%				Studio in azienda
%-------------------------------------------------------------

\mcchap{Fase di formazione aziendale}{cap:agile}

Durante i primi due mesi di tirocinio, la maggior parte del mio tempo è stato speso in formazione,
l'obbiettivo dello studio era di fornirmi le conoscenze adeguate per lavorare nel progetto, in particolare lo studio
di Wordpress, e per integrarmi al Team Iguana per lo sviluppo e la manutenzione di Kiruby.


Il mio studio può essere diviso in tre argomenti principali:
\begin{itemize}
\item Ruby e le sue librerie
\item Wordpress
\item Sviluppo agile
\end{itemize}

Lo studio dei primi due è stato prevalentemente pratico, mentre lo studio e l'applicazione
delle metodologie agili è stato anche in buona parte teorico.

\section{Ruby e le sue librerie}

Il primo argomento di studio una volta entrato in azienda è stato il linguaggio \emph{Ruby}, essendo questo il
linguaggio del web server \emph{Kiruby} la quasi totalità del lavoro del mio Team viene svolto con questo linguaggio.

Per lo studio di Ruby mi sono affidato esclusivamente a risorse disponibili online, consigliate dal Team,
in particolare il sito \url{rubylearning.com}\cite{RUBY}.

In contemporanea a questo ho messo subito in pratica gli argomenti imparati risolvendo i \emph{Koans}\cite{KOANS}
di Ruby, ovvero una serie di test divisi per argomenti, come ad esempio String, Symbols, Hashes, Regular Expression, 
che inizialmente falliscono e che vanno sistemati in modo da passare correttamente. Quindi una volta studiato un certo argomento
veniva risolto il suo rispettivo \emph{Koan}. 

Una volta apprese le basi del linguaggio, per prendere maggiore confidenza ho iniziato a sviluppare
vari Kata agili.

I Kata agili sono dei programmi da sviluppare iterativamente, ovvero prima vengono fornite delle specifiche, una volta 
implementate se ne aggiungono di nuove o si modificano le precedenti. 

Questo tipo di esercizi serve a far abituare
lo svilppatore a scrivere codice in modo manutenibile, ovvero in modo che l'aggiunta o il cambiamento di una qualche 
funzionalità richieda il minimo sforzo grazie alla qualità del codice scritto.

Oltre alle basi del linguaggio ed ai Kata agili ho studiato anche l'utilizzo di alcune \emph{gemme} (così vengono chiamate le 
librerie Ruby) che vengono
usate frequentemente in Kiruby tra cui Test-Unit come framework di testing, librerie per l'integrazione
con SolR e Redis e librerie per lo scambio di dati JSON attraverso HTTP come \emph{HTTPParty}.

\section{Wordpress}

Parallelamente allo studio di Ruby mi sono dedicato allo studio di Wordpress.

Wordpress è il Content Managemente System più diffuso al mondo, è un applicazione web scritta in PHP
che deve essere servita da un web-server, solitamente Apache, nel nostro caso Nginx.

Nella fase di studio mi sono concentrato nell'analizzare e studiare come è struttrata l'applicazione
e dove un programmatore deve mettere mano per fare modifiche e personalizzare la propria installazione.

Oltre allo studio di Wordpress ho fatto un approfondimento del linguaggio PHP del quale avevo qualche
nozione ed esperienza di utilizzo, non troppo approfondita, già prima del mio arrivo. 




%-------------------------------------------------------------
%				AGILE
%-------------------------------------------------------------

\section{Sviluppo agile}

In 7Pixel si sviluppa utilizzando metodologie agili: il software viene sviuppato
iterativamente, nuove feature e aggiornamenti vengono pubblicati quotidianamente.
Nel periodo di studio le varie esercitazioni sono state effettuate
utilizzando due metodologie tipiche dello sviluppo agile.

\begin{itemize}
\item Test Driven Development
\item Pair Programming
\end{itemize}

\subsection{Test Driven Development}
Il TDD Test Driven Development è una tecnica utilizzata ovunque in azienda per lo sviluppo, nel TDD
prima di aggiungere una qualsiasi nuova funzionalità si scrive un test che passerebbe solo se 
quella funzionalità fosse implementata correttamente. 

Una volta scritto il test che fallisce, si cerca nel modo più veloce e semplice possibile di fare passare il test.

Una volta implementato il codice per far passare il test si fa del refactoring per rendere il codice più leggibile
e soprattutto manutenibile, cercando di eliminare il più possibile la presenza di codice duplicato.

\subsection{Pair Programming}
Pair Programming significa svilippo in coppia, ovvero due membri del Team lavorano contemporaneamente
allo stesso codice sulla stessa macchina, utilizzando due schermi speculari, due tastiere e due mouse

Il Pair Programming si rivela molto efficace, perchè si riescono ad evitare molti
errori di distrazione che possono costare caro in termini di tempo e soprattutto
si ha molto spesso la possibilità di confrontarsi con punti di vista diversi che possono
portare ad un analisi più approfondita del problema e a soluzioni migliori. 

Dal mio punto di vista di apprendista il Pair Programming ha portato inoltre il vantaggio
di poter lavorare con gente più esperta e quindi, durante il lavoro, di imparare tecniche,
metodologie e \emph{best pracitces} per lo sviluppo.

\subsection{Studio teorico}
Oltre alla messa in pratica delle tecniche di sviluppo agile sono stati anche effettuati studi teorici
su le come sviluppare codice pulito e manutenibile.

Gli argomenti principali sono stati:
\begin{itemize}
\item Studio dei principali design pattern, tra cui i pattern GRASP e SOLID.
\item Metodologie di refactoring.
\item Metodologie per mantenere il codice ordinato e leggibile.
\end{itemize}
