
%-------------------------------------------------------------
%				Studio in azienda
%-------------------------------------------------------------

\mcchap{Fase di formazione aziendale}{cap:agile}

Durante i primi due mesi di tirocinio, la maggior parte del mio tempo è stato speso in formazione,
il mio studio può essere diviso in tre argomenti principali:

\begin{itemize}
\item Tecnologie
\item Sviluppo agile
\end{itemize}

Lo studio dei primi due è stato prevalentemente pratico, mentre per lo sviluppo
agile lo studio è stato in gran parte teorico ma anche pratico

\section{Ruby e suoi framework}

All'inizio il linguaggio

Poi kata agili (con framework di testing)

Poi redis and co


\section{Wordpress}


%-------------------------------------------------------------
%				AGILE
%-------------------------------------------------------------

\section{Sviluppo agile}

In 7Pixel si sviluppo utilizzando metodologie agili, il software viene sviuppato
iterativamente, nuove feature e aggiornamenti vengono pubblicati quotidianamente.
Nel periodo di studio 

\subsection{TDD}
Una tecnica rigorosamente usata in azienda è il TDD Test Driven Development ovvero, prima di
aggiungere una qualsiasi nuova funzionalità si scrive un test che passa solo se 
quella funzionalità fosse implementata. 

Dopodichè si cerca nel modo più veloce e semplice possibile di fare passare il test

Una volta passato il test si fa del refactoring per rendere il codice più leggibile
e soprattutto manutenibile

\subsection{Pair Programming}
Tutti i lavori effettuati su Kiruby sono sempre stati fatti in Pair Programming
con un membro del team

Il Pair Programming si rivela molto efficace, perchè si riescono ad evitare molti
errori "di distrazione" che possono costare caro in termini di tempo e soprattutto
si ha molto spesso la possibilità di confrontarsi con punti di vista diversi che possono
portare ad un analisi più approfondita del problema e a soluzioni migliori. 

Dal mio punto di vista di apprendista il Pair Programming ha portato inoltre il vantaggio
di poter lavorare con gente più esperta e quindi, durante il lavoro, di imparare tecniche 
metodologie e "best pracitces" per lo sviluppo.
