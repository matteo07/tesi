
%-------------------------------------------------------------
%			PAGE BUILDER
%-------------------------------------------------------------

\mcchap{Integrazione Page Builder}{cap:pb}

Per passare ad una gestione più semplice delle pagine da parte dei content
mi è stato chiesto di cercare qualche sistema che permettesse una suddivisione
della pagina in componenti facilmente editabili e riutilizzabili.-

Le esigenze erano:
\begin{itemize}
\item modificare il contenuto delle componenti da interfaccia grafica e non 
editando il codice HTML.
\item poter creare semplicemente copie delle componenti già create
\item poter spostare le varie componenti con \emph{drag and drop} e avere
feedback visivo immediato della modifica della pagina
\end{itemize}

Per questo mi è stato consigliato di fare una ricerca tra le varie soluzioni disponibli
nella ampia libreria di plugin di Wordpress.

Tra queste è stata individuata una soluzione open source chiamata \emph{Page Builder by
Site Origin} che soddisfava le esigenze.

Il plugin \emph{Page Builder} permette la suddivisione della pagina in colonne all'interno delle quali
si possono inserire delle componenti standard fornite dal plugin oppure dei \emph{Widget}.
Queste componenti possono essere duplicate, modificate aprendo il form della componente e spostate
con drag and drop.
\section{Widget}
I \emph{Widget} sono delle componenti riutilizzabili che possono essere aggiunte a qualsiasi
tipo di pagina Wordpress. Per creare un widget bisogna implementare una sottoclasse di WpWidget
e sovrascrivere il metodo \emph{innin} dove viene restituito il contenuto HTML che la pagina deve restituire
e, se si vuole del contenuto dinamico, sovrascrivere \emph{form} dove viene creato il form che verrà visualizzato dai content per modificare le
parti dinamiche della componenti utilizzato da \emph{innin}.

\section{Inizializzazione}
